\section{Operating principles}

Having successfully constructed the DNA nanopiston, the operation cycle can now be
discussed. For convenience, we take the rotaxane-ds configuration as the starting point
of this cycle. The power stroke of the molecular machine is initiated by bringing the
appropriate chemical fuel, ssDNA $4$ $(0.5\ \mu M)$, into solution at the trans-side. The
DNA strand is fully complementary to ssDNA $2$, thereby inducing a toehold-mediated
strand displacement. Here the flexible overhang located at the end of ssDNA $2$, referred
to as the toehold, is used to mediate the hybridisation of ssDNA $2$ and $4$ (Fig ..).

During the strand displacement reaction, it is hypothesised that different transient
states can possibly occur. On of the possibly scenario's describes the hybridisation
happening inside of the nanopore. This scenario is deemed to be unlikely, since this
process would require three strands of ssDNA to be simultaneously present inside the
constriction of the nanopore. Alternatively, the hybridisation can take place outside of
the nanopore, in the trans-side of the reservoir. This process implies that the
neutravidin protein would enter the lumen of the pore, which has been showed to be
possible by previous studies. This variation of the transient state is thereby
thought of as the most probable.

The resulting configuration is called the rotaxane-ss in view of the fact that it is
predominately composed of ssDNA. During this process a DNA duplex, composed of the ssDNA
2 and 4, is released into the trans-side of the reservoir.

Subsequently the cargo strand, $0.5\ \mu M$ of ssDNA $2$,  is brought into solution at
the cis-side, inducing the piston's recovery stroke. In this process the cargo hybridises
with the rotaxane-ss, re-establishing a rotaxane-ds structure and completing the cycle.
Each piston interaction transports one cargo strand, from the cis- to the trans-side of
the nanopore, turning over one fuel strand in the process.

Important to note is that no external potential is specified for operating the DNA
nanopiston. In contrast to earlier DNA transporters, the piston is able to function in a
range of applied transmembrane biases. Experimentally it is verified that the
cycle operates at positive, $+20\ mV$, $+50\ mV$ and $+100\ mV$ and negative, $ - 20\
mV$. The limited range observed for negative voltages is most likely resulting from the
inability of the fuel strand to hybridise with the toehold of rotaxane-ds. This
hybridisation reaction is a rate limiting process, resulting in faster cycles at positive
then at negative applied bias shown in Fig .... Other factors, like the accesiblity of
cargo strands during the hybridisation of rotaxane-ss also influence the cycle rate.


The ability of the nanopiston to transport cargo both with and against an external bias
is an important property of this molecular machine. It suggests that the externally
applied bias might not be necessary for its functioning. Suspected is that the entropic
interactions between the DNA strands and the nanopore are expected to play an important
role in this behaviour. To accurately analyse these effects, further analysis is needed.
\begin{figure}[ht!]
  \begin{centering}
  \adjustbox{minipage=1.3em,valign=t}{\subcaption{}\label{sfig:testa}}%
  \begin{subfigure}[t]{\dimexpr.95\linewidth-1.3em\relax}
  \centering
  \includegraphics[width=\linewidth,valign=t]{Figures/FluctuationRotaxane.png}
  \end{subfigure}%
  \vspace{0.6cm}
  \adjustbox{minipage=1.3em,valign=t}{\subcaption{}\label{sfig:testb}}%
  \begin{subfigure}[t]{\dimexpr.5\linewidth-1.3em\relax}
  \centering
  \includegraphics[width=\linewidth,valign=t]{Figures/RotaxaneCycle.png}
  \end{subfigure}
  \caption{This is a figure \newline \newline \newline \newline \newline}
  \label{fig:test}
  \end{centering}
\end{figure}

\newpage

%  ⡏⢱ ⠄ ⣰⡀   ⣀⡀ ⢀⡀ ⢀⡀   ⢀⡀ ⡀⣀ ⢀⡀ ⢀⡀ ⣀⡀ ⢀⣀   ⣇⡀ ⠄ ⠠   ⢀⣀ ⢀⣀ ⣇⡀ ⡀⣀ ⠄ ⠠ ⡀⢀ ⢀⡀ ⣀⡀
%  ⠧⠜ ⠇ ⠘⠤   ⠇⠸ ⠣⠜ ⣑⡺   ⠣⠭ ⠏  ⣑⡺ ⠣⠭ ⠇⠸ ⠭⠕   ⠧⠜ ⠇ ⡸   ⠭⠕ ⠣⠤ ⠇⠸ ⠏  ⠇ ⡸ ⠱⠃ ⠣⠭ ⠇⠸
% -----------------------------------------------------------------------------------------

% misschien bij de biological nanopores.
% Electrophoretic, electro-osmotic and entropic forces are, in principle, acting on the
% rotaxanes. The electrophoretic force sets the negatively charged DNA in motion, under the
% action of the applied bias (from cis to trans for
% ∆?? > 0 ). Electroosmosis generates an opposing force, arising from the motion of cations
% accumulated on the walls of the negatively charged ClyA pore and the DNA thread. Finally,
% the entropic force is solely geometry specific, and pushes the rotaxane towards
% conformations with high configurational entropy. Entropic forces are expected to play an
% important role in the rotaxanes studied here, which are composed of stiff dsDNA and
% flexible ssDNA parts.
