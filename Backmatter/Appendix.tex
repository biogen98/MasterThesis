\chapter{One Dimensional Confined Diffusion}

This appendix provides a detailed description of the motion of a Brownian particle
confined to a one dimensional domain. Due to the statistical nature of this motion we
will discuss the evolution of the probability density function $\phi(x,t)$, which
represents the probability of finding the particle on the position $x$ at time t.
Once the evolution of this probability function is known, the statistical properties can
be evaluated. Here we will discuss the mean square displacement (MSD) as it is used to
analyse the motion of the fully double stranded mixed rotaxane.\\
A central result in statistical mechanism is that the evolution of
$\phi$ is  described by the diffusion equation, given by
\begin{equation}
  \frac{\partial \psi}{\partial t} =  D \frac{\partial^2 \psi}{\partial x^2}, \quad
  \label{eq:diff}
\end{equation}
here D is the diffusion coefficient of our particle.
The confinement is imposed through reflecting boundary conditions at $x=0$ and $x=L$,
this is equivalent to imposing a
vanishing particle current at the boundaries of our domain, $j = - D \frac{\partial
\psi}{\partial x} = 0$. Solving the heat the equation can be done using the method of
separation of variables, here we assume that the solution can be expressed as $ \psi(x,t)
= f(x)g(t)$. Upon substitution \ref{eq:diff} becomes,
\begin{equation}
  \frac{\dot{g}}{g} = \frac{\ddot{f}}{f} = - \alpha,
\end{equation}
here the two expressions are implied to be constant since the variables are independent.
The partial differential equation is now treated as two seperate ordinary differential
equations, for which we find,

\begin{align}
t:\quad \dot{g} = - \alpha g(t) \Rightarrow g(t) = e^{-\alpha t},
\end{align}

\begin{align}
  x:\quad D \ddot{f} = - \alpha f(x) \Rightarrow f(x) &= A \sin(K x) + B \cos(Kx)\\
  &= B \cos(\frac{\pi n x}{L}).
\end{align}
In the latter expression the boundary conditions impose $A=0$ and a constraint on the
parameters given by the relation,
\begin{align}
  \frac{\alpha}{D} = \frac{\pi^2 n^2}{L^2}.
\end{align}
By substituting the found results into the assumed form of the solution, we find the
general solution to the confined diffusion equation as the linear combination,
\begin{align}
  \psi(x,t) &= \sum_{n=0}^{+\infty} C_n \cos\Big(\frac{\pi n x}{L}\Big) e^{- \frac{D\pi^2
  n^2}{L^2}t}.
\end{align}

At time $t=0$ the particle is found at set at $x_0$ resulting in the initial condition,
\begin{align}
  \psi(x, 0) = \delta(x-x_0) = \sum_{n=0}^{+ \infty} C_n \cos(\frac{\pi n x}{L}).
\end{align}
Imposing this initial condition on the found general solution of the confined diffusion
equation gives,
\begin{align}
  \psi(x, t)=\frac{1}{L} \Bigg[ 1 + \sum_{n=1}^{+\infty} \cos\Big(\frac{\pi n
  x_0}{L}\Big) \cos\Big(\frac{\pi n x}{L}\Big) e^{- \frac{D\pi^2  n^2}{L^2}t}\Bigg].
\end{align}
This expression describes the behaviour of a Brownian particle in a one dimensional
confined domain. Using the found expression the MSD is calculated to be,
\begin{align}
  \langle \Delta x^2 \rangle &= \langle(x-x_0)^2\rangle\\&= \frac{L^2}{6}\Bigg[1 -
  \frac{96}{\pi^4}
  \sum_{k=0}^{+\infty} \frac{1}{(2k+1)^4} e^{- \frac{D(2k+1)^2 \pi^2}{L^2}t}\Bigg].
\end{align}
As expected, the mean squared distances saturates to $\langle \Delta x^2 \rangle = L^2/6$
in the long-time limit $t \gg L^2 / D.$ To explore the other limiting case $t \ll L^2/D
$ we perform a Taylor expansion of the exponential and find,
\begin{align}
  \langle \Delta x^2 \rangle &= \frac{L^2}{6} - \frac{16 L^2}{\pi^4} \sum_{k=0}^{\infty}
  \frac{1}{(2k+1)^4} + \frac{16 D t}{\pi^2} \sum_{k=0}^{\infty} \frac{1}{(2k+1)^2} +
  \mathcal{O}\bigg(\frac{D^2 t^2}{L^4}\bigg).
\end{align}
Using the two convergent series,
\begin{equation}
  \sum_{k=0}^{\infty} \frac{1}{(2k+1)^2} = \frac{\pi^2}{8} \qquad \text{and} \qquad
  \sum_{k=0}^{\infty} \frac{1}{(2k+1)^4} = \frac{\pi^4}{96}
\end{equation}
the free diffusion is recovered at short time scales, \cite{BICKEL200724}
\begin{equation}
\langle \Delta x^2 \rangle = 2Dt \qquad for\,\, t \ll L^2/D.
\end{equation}
