\chapter{One Dimensional Confined Diffusion}

This appendix provides a detailed description of the motion of a Brownian particle
confined to a one dimensional domain. Due to the statistical nature of this motion we
will discuss the evolution of the probability density function $\phi(x,t)$, which
represents the probability of finding the particle on time t at the position $x$ in our
one dimensional domain.  Once the
evolution of this probability function is known, the statistical properties can be
evaluated. Here we will discuss the mean square displacement (MSD) as it is used to
analyse the motion of the fully double stranded mixed rotaxane.

A central result in statistical mechanism is that in the continuum limit the evolution of
$\phi$ is  described by the heat equation, given by
\begin{align*}
  \frac{\partial \psi}{\partial t} =  D \frac{\partial^2 \psi}{\partial x^2}, \quad
\end{align*}
where D is the diffusion coefficient of our particle.
The Confinement is imposed through reflecting boundary conditions $j = - D \frac{\partial
\psi}{\partial x} = 0$. This condition enforces the vanishing particle current at the
boundaries. Solving the heat the equation is often times done by performing a separation
of variables, here we express $ \psi(x,t) = f(x)g(t)$.

As a result we find,
\begin{align*}
t:\quad \dot{g} = - \alpha g(t) \Rightarrow g(t) = e^{-\alpha t}
\end{align*}

\begin{align*}
  x:\quad D \ddot{f} = - \alpha f(x) \Rightarrow f(x) &= A \sin(K x) + B \cos(Kx)\\
  &= B \cos(\frac{\pi n x}{L})
\end{align*}
where the boundary conditions impose the following restrictions
\begin{align*}
  \frac{\alpha}{D} = \frac{\pi^2 n^2}{L^2}
\end{align*}

The general solution is given by the linear combination,
\begin{align*}
  \psi(x,t) &= \sum_{n=0}^{+\infty} C_n \cos\Big(\frac{\pi n x}{L}\Big) e^{- \frac{D\pi^2
  n^2}{L^2}t}\\
            &=\frac{1}{L} \Bigg[ 1 + \sum_{n=1}^{+\infty} \cos\Big(\frac{\pi n
  x_0}{L}\Big) \cos\Big(\frac{\pi n x}{L}\Big) e^{- \frac{D\pi^2  n^2}{L^2}t}\Bigg]
\end{align*}
\begin{align*}
  \langle \Delta x^2 \rangle &= \langle(x-x_0)^2\rangle\\&= \frac{L^2}{6}\Bigg[1 -
  \frac{96}{\pi^4}
  \sum_{k=0}^{+\infty} \frac{1}{(2k+1)^4} e^{- \frac{D(2k+1)^2 \pi^2}{L^2}t}\Bigg]\\
\end{align*}
As expected, the mean squared distances saturates to $\langle \Delta x^2 \rangle = L^2/6$
in the long-time limit $t \gg L^2 / D.$ To explore the other limiting case $t \ll L^2/D
$ we perform a taylor expansion and find
\begin{align*}
  \langle \Delta x^2 \rangle &= \frac{L^2}{6} - \frac{16 L^2}{\pi^4} \sum_{k=0}^{\infty}
  \frac{1}{(2k+1)^4} + \frac{16 D t}{\pi^2} \sum_{k=0}^{\infty} \frac{1}{(2k+1)^2} +
  \mathcal{O}\bigg(\frac{D^2 t^2}{L^4}\bigg).
\end{align*}

\cite{BICKEL200724}

\begin{equation*}
  \sum_{k=0}^{\infty} \frac{1}{(2k+1)^2} = \frac{\pi^2}{8} \qquad \text{and} \qquad
  \sum_{k=0}^{\infty} \frac{1}{(2k+1)^4} = \frac{\pi^4}{96}
\end{equation*}

\begin{equation*}
\langle \Delta x^2 \rangle = 2Dt \qquad t \ll L^2/D
\end{equation*}
