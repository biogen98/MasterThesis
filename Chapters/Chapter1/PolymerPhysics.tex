\section{Polymer Physics}
A polymer is a biomolecule made up of building blocks called monomers, linked together to
from a chain. The configuration of this chain is determined by the position vector of
each monomer,denoted as $\{\boldsymbol{r}_0, \boldsymbol{r}_1, \dots,
\boldsymbol{r}_N\}$.  The link between each consecutive pair of monomers is called the
bond-vector, defined as
$\boldsymbol{u}_i = \boldsymbol{r}_i - \boldsymbol{r}_{i-1}$. During this discussion we
will assume these bonds to be inextensible, i.e. having a fixed
bond length $|\boldsymbol{u}_i| = a$.\\


Various different model can be used to model a polymer, the most simple one is called the
Freely Jointed Chain (FJC). This model is an example of an ideal flexible polymer, in
which no excluded volume interactions or polymer bending is taken into account. In this
model it is assumed that each bond-vector is completely uncorrelated with its adjacent
bonds. Mathematically this is represented by assigning the bond-vector orientation using
the uniform distribution
\begin{equation}
    g(\boldsymbol{u}) = \frac{1}{4 \pi a}
    \delta(|\boldsymbol{u}| - a), \\
\end{equation}
where a is the fixed bond length.


The above described model provides a relatively accurate description of long polymers,
however the assumption that consecutive monomers are uncorrelated becomes
problematic at small length scales . The Kratky-Porod, or discrete wormlike chain, model
solves this problem by taking the energetic cost of bending the polymer into
account. Mathematically this is done introducing a bending rigidity between neighbouring
bonds in the form of a coupling constant, $\kappa >0$. Each polymer configuration is
assigned an energy using the equation,
\begin{equation}
    E_{WLC}= -\kappa \sum_{i=1}^{N} \boldsymbol{\hat{u}_i} \cdot
    \boldsymbol{\hat{u}}_{i+1}
    = -\kappa
    \sum_{i=1}^{N} \cos\theta_i,
    \label{wlc}
\end{equation}
where $\boldsymbol{\hat{u}} = \boldsymbol{u}/a$ is the unit bond-vector and $\theta_i$
the angle between the  bond-vectors $\boldsymbol{\hat{u}_i}$ and
$\boldsymbol{\hat{u}_{i+1}}$. The lowest energy state of this discrete wormlike chain is
a straight rodlike configuration, where the bond angles $\theta_i$ are minimized.

To calculate the bond-vector correlation function,  we first determine the partition
function, $Z$, of the system. Identifying the single monomer contributions, the partition
function factorises into a product of single bond-vector partition as,
\begin{equation}
    \begin{aligned}
        \label{210}
        Z_{\mathrm{WLC}}(N, T)
        &= \int_{0}^{\pi}\dots \int_{0}^{\pi} d \theta_1 \dots d \theta_N \sin \theta_1 \dots
        \sin \theta_N\ e^{\beta \kappa \sum_{i=1}^{N-1} \cos\theta_i}\\
        &= \left[\int_{0}^{\pi} d \theta \sin \theta e^{\beta \kappa \cos
        \theta}\right]^{N}\\
        &= \left[Z_{\mathrm{WLC}}(1, T)\right]^{N},
    \end{aligned}
\end{equation}
where $\beta=1/\kappa_b T$ is the inverse temperature. It rests us to determine the
single bond-vector partition function. Carrying out the integration yields the result,
\begin{equation}
    Z_{\mathrm{WLC}}(1, T)=\int_{0}^{\pi} d \theta e^{\beta \kappa \theta}=\frac{2
    \sinh(\beta \kappa)}{\beta \kappa}.
\end{equation}

%\begin{equation}
%    \left\langle\cos \theta_{i+1}\right\rangle
%    =\frac{\partial \log Z_{\mathrm{WLC}}(1, T)}{\beta \partial \kappa}.
%\end{equation}

From the found partition function we can now determine the bond-vector correlation
function. Using the definition of the partition function, we determine the average cosine
of the angle between consecutive bonds to be,
\begin{equation}
    \begin{aligned}
    \left\langle\cos \theta_{i+1}\right\rangle
    &=\frac{\partial \log Z_{\mathrm{WLC}}(1, T)}{\partial(\beta \kappa)}\\
    &= \frac{1}{\tanh(\beta \kappa)} - \frac{1}{\beta \kappa}.
    \end{aligned}
\end{equation}

Studying the conformation of polymers is often times done by working in the  at low
temperatures or with a large bending rigidity, $\kappa$, we find
that the above expression simplifies. In the limit, $\beta \kappa \gg 1$, the lowest
order approximation yields,
\begin{equation}
    \langle\cos \theta\rangle \approx 1-\frac{1}{\beta \kappa}.
\end{equation}

Decomposing the bond-vector $\boldsymbol{\hat{u}}_{n-1}$ in terms an orthonormal basis
defined by the normal and tangential directions of the preceding vector
$\boldsymbol{\hat{u}}_{n-1}$ gives
\begin{equation}
\boldsymbol{\hat{u}}_{n+1} = \boldsymbol{\hat{u}}_{n} \cos \theta_{n} +
\boldsymbol{\hat{u}}_{n}^{\perp} \sin \theta_{n}.
\end{equation}
This decomposition allows us to express the correlation between distant bond-vectors in
terms of the correlation between neighbouring bonds. Performing the factorisation yields
\begin{equation}
\begin{aligned}
    \left\langle\boldsymbol{\hat{u}}_{i} \cdot \boldsymbol{\hat{u}}_{i+m}\right\rangle
    &=\left\langle\boldsymbol{\hat{u}}_{i} \cdot
        \boldsymbol{\hat{u}}_{i+m-1}\right\rangle\left\langle\cos
    \theta\right\rangle = \dots =\langle\cos \theta\rangle^{m},
    % &=\exp \bigg{[} -\frac{n a}{l_{b}} \bigg{]},
\end{aligned}
\end{equation}
here we used the fact that the sinusoidal terms vanish due to symmetry.
Exploring this result in the limit, $\beta \kappa \gg 1$, we find the expression
\begin{equation}
    \left\langle\boldsymbol{\hat{u}}_{i} \cdot \boldsymbol{\hat{u}}_{i+m}\right\rangle =
    e^{m \log(1 - \frac{1}{\beta \kappa})} \approx e^{-na/l_p},
\end{equation}
introducing a new polymer quantity, the bending persistence length

\begin{equation}
    l_b \equiv \frac{a \kappa}{k_{b} T}.
\end{equation}
This general result in polymer physics states that the correlations between bond-vectors
is exponentially decreasing. The defined quantity represents the characteristic
length scale of the polymer over which the correlations between
bond-vectors is lost.

Two limiting cases can be explored, firstly in the
case where the persistence length is much larger then the polymer's length, $l_p \gg na$,
all bond-vectors are correlated, i.e. the polymer approximates a straight rod. The
reverse case where $l_p \gg na$, it can easily be shown that the polymer behaves as a
stochastic random walk.

The persistence length is a central result in the theory of polymer physics, providing a
measurable quantity related to the bending rigidity of of a polymer. During the
simulations performed in this thesis, the notion of bending persistence length is used to
discuss the flexibility of the DNA polymer.


%The end-to-end vector $\boldsymbol{R}$ in the continuum limit. For example can be
%rewritten using the arc-length parameter $s$ where $0 \leq s \leq L$ and $L = Na$ as:
%\begin{equation}
%    \label{hoi}
%    \langle\widehat{u}(q) \cdot \widehat{u}(q+s)\rangle= e^{-s / l_{\mathrm{p}}}.
%\end{equation}
%The continuum version of the end-to-end vector
%\begin{equation}
%    \boldsymbol{R}=\int_{0}^{L} \widehat{u}(s) d s.
%\end{equation}

%\begin{equation}
%\begin{aligned}
%    \left\langle\boldsymbol{R}^{2}\right\rangle
%    &= \int_{0}^{L} d s d s^{\prime}\left\langle\widehat{u}(s) \cdot
%  \widehat{t}\left(s^{\prime}\right)\right\rangle \\
%    &= 2 l_{\mathrm{b}} L\left\{1-\frac{l_{\mathrm{b}}}{L}\left(1-e^{-L /
%l_{\mathrm{b}}}\right)\right\}.
%\end{aligned}
%\end{equation}

