\section{Thesis outline}

All organisms in nature tirelessly perform work, struggling against an ever increasing
entropy. This work is collectively performed by countless molecular machines, all
contributing to their specific tasks.

Despite being so abundantly present in nature, fabricating synthetic molecular machines
turns out to be a difficult task. One of the biggest hurdles in this process arises from
the length-scale of these machines. Often times these structures are not larger then
a few nanometres, making the typical energy associated with the bonds and
distortions of their structure comparable to thermal energy fluctuations.

As a result, the random thermal fluctuations in the environment of these molecular
machines induce the stochastic motion that complicates their functioning. Extracting
useful work from this Brownian motion, performed by the freely tumbling structures, is
almost impossible. To overcome this limitation, most synthetic molecular machines are
embedded in a larger complex providing necessary structure.

This phenomenon is also observed in nature, for instance in the interfacing of protein
complexes with the phospholipid bilayer of cells.  A wildly known example is the
bacterial flagella motor, which provides an efficient way for bacteria to
roll and tumble through their environment. Just Like in electrical motors, the flagella
consists of a stator and a rotor. The stator is anchored into the cell membrane, while
the rotor is allowed to freely rotate. The work is produced by the flow of cations
through the stator, inducing changes in the electrostatic interactions between the two
parts of the flagella, generating unidirectional motion.\\

Analogues to macroscopic engines, during the operation of molecular machines, heat is
produced. When the structure is not capable of dissipating this heat efficiently, an
excessive build up limits the life cycle of the complex. To mitigate this problem, often
times large and soft molecules are used in the design of nanomachines. A logical choice
is the use of polymers, which can effectively dissipate heat as a result of their
flexibility.  Due to the programmability of DNA, using the Watson-Crick interactions,
the DNA polymer provides additional aptitude, making it a popular material in
nanotechnology.

The central topic of this thesis is studying the DNA nanopiston, designed by Bayoumi et
al., a DNA based molecular machine embedded into a phospholipid membrane. This nanopiston
can
be characterised as an autonomous molecular machine, which turns over chemical fuel to
continuously perform work. The aim of this complex is to perform selective transport of
DNA through a membrane. The operation cycle of previously designed DNA transporters
require a supporting external bias, where the nanopiston operates also against an
external bias. The physics driving this machine is entropy and will be discussed in
detail in this thesis.

In the first chapter, an introduction is given into important concepts regarding the
DNA nanopiston. Having laid this theoretical foundation, the structure and operation
cycle of the DNA nanopiston are discussed in chapter two. Next the simulation model used
in this thesis is presented in chapter three. The results of these simulations are
finally discussed in the chapter four. The final chapter five of this thesis offers a
discussion of the results and recommendations for further research.

\begin{figure}[ht]
\begin{center}
  \includegraphics[width=0.85\textwidth]{Figures/flagella2.png}
  \caption{Flagella motor}
\end{center}
\end{figure}
