\noindent In this work the novel molecular machine devised by Bayoumi et al. has been
studied using
molecular dynamics simulations. The primary objective of this research was to shed
light on the operating principles facilitating the autonomous and active transport which
characterise this DNA nanopiston. In this chapter the obtained results are summarized,
after which recommendations for future research are made.

\section{Results \& Conclusions}

Understanding the underlying processes driving the operation of molecular machines is an
important corner stone in furthering their development. As a result of the scale and
complexity of these structures, shedding light on these interactions is experimentally
challenging. In this work we aimed to provide an insight into the operation of the
DNA nanopiston using molecular dynamics simulations.

This thesis builds further upon simulations presented in the original paper, where a
coarse-grained model of the DNA nanopiston was developed based on both theoretical and
experimental considerations. The central component of this model is the DNA rotaxane
which was simulated using a bead-and-spring model. To increase the scope of the research,
we designed a new coarse-grained model of the DNA nanopiston utilising a more accurate
representation of DNA, namely OxDNA. This new model yields a more realistic description
of DNA while also enabling us to simulate DNA hybridisation reactions.

The entropic penalty of confining single stranded DNA inside of the nanopore is presumed
to play an important role in the mechanisms driving the DNA piston. We studied these
entropic effects by simulating a specifically engineered class of rotaxanes, called the
mixed rotaxanes. These simulations indicate that there are two competing entropic
interactions present in the rotaxane-pore complex. A first entropic force is found to
arise from confining the large dsDNA strands of the rotaxane inside of the pore.
Competing with this force are the smaller but more flexible ssDNA strands which endeavour
to maximize their available configurational space by opposing the confinement of the
pore. As the length of these ssDNA strands is increased, this latter interaction starts
to dominate.

After establishing this essential understanding of the entropic interactions, the two
stable states of the piston's operating cycle are simulated, i.e. rotaxane-ss and
rotaxane-ds. These simulations indicated that the sequestering of the overhang of
rotaxane-ss is prevented by the entropic penalty of confining it in the pore.
The established entropic force enables the hybridisation with fuel strands even opposing
an upward electrophoric force, differentiating the DNA nanopiston as an active
transporter. Due to the rotaxane geometry this hybridisation reaction results in the
formation of rotaxane-ss in a low-entropy state. As a consequence the rotaxane-ss
spontaneously migrates in the trans-cis direction exposing the flexible ssDNA strand to
the cis-reservoir, enclosing the dsDNA fraction inside of the pore. This
upward motion plays an important role in exposing the ssDNA fraction of rotaxane-ss to
the cis-reservoir, where it can hybridise with a cargo strand. Here we concluded that
the interplay of the hybridisation reactions and the entropic interactions collectively
drive the molecular machine.

In the last part of this thesis an attempt was made to study the ensemble of
hybridisation pathways providing the free energy to drive the nanopiston. The study of
these thermodynamic transitions was complicated by the complex reaction kinetics and an
initial energy barrier. To overcome this difficulty an forward flux sampling algorithm
was used to analyse these transitions. From the performed simulations we concluded that
the static representation of the ClyA pore in our coarse-grained model inhibits these
hybridisation reactions. The compliance of the pore constriction is found to be essential
in facilitating the hybridisation reactions, yet were not incorporated in our model.

The performed computational analysis of the DNA nanopiston confirmed earlier research on
the operating principles facilitating the piston cycle. The exploratory research that was
performed motivates further research towards the functioning of this nanopiston.
Eventually, this knowledge will aid the development of novel molecular devices, further
blurring the line between life and the artificial.
