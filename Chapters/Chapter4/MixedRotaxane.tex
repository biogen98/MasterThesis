\section{Mixed Rotaxane}

%% Zelf nog schrijven
% To highlight the crucial role of the entropic forces, we designed and studied four
% different rotaxanes, hereby referred to as mixed rotaxanes. These are an alternative
% version of rotaxane-ds, in which the toehold is removed, and the ssDNA connection strand
% varied in length. the total length of the rotaxane thread was fixed at 100 nt/bp, with
% the ss-DNA connection strand length varying from 0 nt to 40 nt and the ds DNA varying
% from 100 bp to 60 bp, in steps of 10 nt/bp
%
% We then performed I-V measurements in the range, see figure.
%
% The mixed rotaxane composed entirely of dsDNA yields an almost linear I-V plot in the
% measured voltage range indicating a constant obstruction of the nanopore. constant
% resistance of the nanopore.
%
% The rotaxane with a 10 nt ssDNA, however, has a rather different behavior. Observed
% blockage at high voltages in experiments.
%
% Longer strands we observe no blockage.
%
% Shed light on these experimental results we performed MD Simulations at zero bias, here
% the entropic contributions can be isolated.
%
% 1D diffusion
%
% We identify the z-axis with the symmetry axis of the pore and fix the origin of the
% coordinate system at the center of the trans-entry of the pore. The center of the cis
% entry is then located at $r_{cis} = (0, 0, z_{0})$ , with respect to the origin,
% where $z_0 = 14.3\ nm$ (see Figure S6). We define the reaction coordinate as follows
%
% \begin{equation}
%   X = \begin{cases}
%         &Z_0 + |\textbf{r} - \textbf{r}_{cis}|, \hspace{0.5cm} \textit{if on cis-side}\\
%         &Z, \hspace{2.5cm} \textit{if inside pore}\\
%         &-|\textbf{r}|, \hspace{2.11cm} \textit{if on trans-side}
%       \end{cases}
% \end{equation}
%
% The simulation data show that the peak gradually shifts towards negative values of $X$
% with increasing ssDNA length, suggesting that full pore blockage requires increasingly
% stronger electrophoretic force.

% To gain some insight on these entropic forces, we performed computer simulations of the
% rotaxane/ClyA complex in absence of an applied potential ($\Delta V = 0$)
%
% We identify the z-axis with the symmetry axis of the pore and fix the origin of the
% coordinate system at the center of the trans-entry of the pore. The center of the cis
% entry is then located at $r_{cis} = (0, 0, z_{0})$ , with respect to the origin,
% where $z_0 = 14.3\ nm$ (see Figure S6). We define the reaction coordinate as follows








\begin{figure}[ht]

  \begin{centering}
  \adjustbox{minipage=1.3em,valign=t}{\subcaption{}\label{sfig:testa}}%
  \begin{subfigure}[t]{\dimexpr.29\linewidth-1.3em\relax}
  \centering
  \vspace{0.6cm}
  \includegraphics[width=1\linewidth,valign=t]{Figures/IV-100.png}
  \end{subfigure}%
  \adjustbox{minipage=1.3em,valign=t}{\subcaption{}\label{sfig:testb}}%
  \begin{subfigure}[t]{\dimexpr.5\linewidth-1.3em\relax}
  \centering
  \vspace{-0.1cm}
  \includegraphics[width=\linewidth,valign=t]{Figures/MR-100.png}
  \end{subfigure}%
  \adjustbox{minipage=1.3em,valign=t}{\subcaption{}\label{sfig:testc}}%
  \begin{subfigure}[t]{\dimexpr.21\linewidth-1.3em\relax}
  \centering
  \vspace{0.3cm}
  \includegraphics[width=\linewidth,valign=t]{Figures/Rotaxane-100.png}
  \end{subfigure}
  \label{fig:test}
  \end{centering}

  \vspace{0.01cm}

  \begin{centering}
  \adjustbox{minipage=1.3em,valign=t}{}%
  \hspace{.35cm}
  \begin{subfigure}[t]{\dimexpr.3\linewidth-1.3em\relax}
  \centering
  \vspace{0.2cm}
  \includegraphics[width=\linewidth,valign=t]{Figures/IV-90.png}
  \end{subfigure}%
  \adjustbox{minipage=1.3em,valign=t}{}%
  \hspace{.25cm}
  \begin{subfigure}[t]{\dimexpr.5\linewidth-1.3em\relax}
  \centering
  \includegraphics[width=\linewidth,valign=t]{Figures/MR-90.png}
  \end{subfigure}%
  \adjustbox{minipage=1.3em,valign=t}{}%
  \hspace{.3cm}
  \begin{subfigure}[t]{\dimexpr.21\linewidth-1.3em\relax}
  \centering
  \vspace{-0.5cm}
  \includegraphics[width=.5\linewidth,valign=t]{Figures/Rotaxane-90.png}
  \end{subfigure}
  \label{fig:test}
  \end{centering}

  \vspace{.01cm}

  \begin{centering}
  \adjustbox{minipage=1.3em,valign=t}{}%
  \begin{subfigure}[t]{\dimexpr.3\linewidth-1.3em\relax}
  \centering
  \vspace{0.2cm}
  \hbox{\hspace{0.35cm}
  \includegraphics[width=1\linewidth,valign=t]{Figures/IV-80.png}}
  \end{subfigure}%
  \adjustbox{minipage=1.3em,valign=t}{}%
  \hspace{-0.5cm}
  \begin{subfigure}[t]{\dimexpr.5\linewidth-1.3em\relax}
  \centering
  \hbox{\hspace{0.51cm}
  \includegraphics[width=\linewidth,valign=t]{Figures/MR-80.png}}
  \end{subfigure}%
  \adjustbox{minipage=1.3em,valign=t}{}%
  \hspace{.5cm}
  \begin{subfigure}[t]{\dimexpr.21\linewidth-1.3em\relax}
  \centering
  \vspace{-0.5cm}
  \includegraphics[width=.6\linewidth,valign=t]{Figures/Rotaxane-80.png}
  \end{subfigure}
  \label{fig:test}
  \end{centering}

  \vspace{.01cm}

  \begin{centering}
  \adjustbox{minipage=1.3em,valign=t}{}%
  \begin{subfigure}[t]{\dimexpr.3\linewidth-1.3em\relax}
  \centering
  \vspace{0.2cm}
  \hbox{\hspace{0.4cm}
  \includegraphics[width=\linewidth,valign=t]{Figures/IV-70.png}}
  \end{subfigure}%
  \adjustbox{minipage=1.3em,valign=t}{}%
  \begin{subfigure}[t]{\dimexpr.5\linewidth-1.3em\relax}
  \centering
  \hbox{\hspace{0.1cm}
  \includegraphics[width=\linewidth,valign=t]{Figures/MR-70.png}}
  \end{subfigure}%
  \adjustbox{minipage=1.3em,valign=t}{}%
  \hspace{0.5cm}
  \begin{subfigure}[t]{\dimexpr.21\linewidth-1.3em\relax}
  \centering
  \vspace{-0.5cm}
  \includegraphics[width=.4\linewidth,valign=t]{Figures/Rotaxane-70.png}
  \end{subfigure}
  \label{fig:test}
  \end{centering}
  \caption{This is a figure.}

\end{figure}



\begin{figure}
\begin{center}
  \includegraphics[width=0.90\textwidth]{Figures/MR-100-diff.png}
  \caption{write caption}
\end{center}
\end{figure}
