\section{Computational setup}

The model used to study the DNA nanopiston, is largely based on the model
previously devised by Bayoumi et al. The main variation between the two models lays in
the different coarse-grained models, used to simulate the DNA strands. As discussed in
Chapter 2, the Bayoumi model uses a bead-spring approach to simulate DNA strands, where
we use a more sophisticated model called oxDNA. This DNA model gives a better
representation of the dynamics of DNA strands, at the same time allowing for accurate
simulations of the thermodynamic transitions in the DNA nanopiston.

The simulations are performed using the popular molecular dynamics simulator, LAMMPS[.].
Employing the Lammps implementation of oxDNA developed by Henri et al[.], it becomes
possible to study the interactions between oxDNA strands and externally defined
particles.
The initial configurations of the simulations are generated using the Moltemplate
package[.], a general purpose molecule builder for LAMMPS.

The molecular dynamics simulations performed in this thesis utilises a langevin
thermostat, more precisely the Dot-C langevin integrator also implemented by Henri et
al. This is a LAMMPS implementation of the “Langevin C” integrator developed by
Davidchack et al. [.], falling in the class of rigid-body Langevin-type integrators.
This type of thermostat separates the stochastic and deterministic parts of a Langevin
thermostat to efficiently take into account the extra degrees of freedom in the system,
arising from the non-spherical shape of the oxDNA beads. As is common practice in MD
simulations, the diffusion coefficient of the oxDNA strand is chosen larger then the
value of physical DNA. This is done to speed up the simulations, while ensuring its
physical accuracy.

The model is used to more accurately study the conformational fluctuations of
the Rotaxane and develop understanding of the entropic interactions between the DNA and
the nanopore. Next the thermodynamic transitions in the operation cycle of the piston are
simulated using a forward flux sampling algorithm. The FFS algorithm is implemented as a
Python script, performing the simulations by interfacing with the Python API of LAMMPS.
