\section{Coarse-grained simulations}

\begin{figure}[ht]
  \begin{centering}
  \adjustbox{minipage=1.3em,valign=t}{\subcaption{}\label{sfig:testa}}%
  \begin{subfigure}[t]{\dimexpr.4\linewidth-1.3em\relax}
  \centering
  \includegraphics[width=0.9\linewidth,valign=t]{Figures/Stefanos1.png}
  \end{subfigure}%
  \adjustbox{minipage=1.3em,valign=t}{\subcaption{}\label{sfig:testb}}%
  \begin{subfigure}[t]{\dimexpr.5\linewidth-1.3em\relax}
  \centering
  \includegraphics[width=0.9\linewidth,valign=t]{Figures/Stefanos2.png}
  \end{subfigure}
  \caption{This is a figure}
  \label{fig:test}
  \end{centering}
\end{figure}

% to investigate the typical conformations of rotaxanes-ds and ss at zero bias. This
% approach is justified by the long time scales ofinterest, and the high salt concentration
% of the solution (2 M KCl). A numerical analysis ofthe electrostatic energies, using the
% Poisson−Boltzmann equations, indeed indicates that electrostatic DNA-nanopore
% interactions are weak and repulsive (Figures S4 and S5 in Section S8). These interactions
% were accounted for by properly adjusting the dimensions of the pore
%
% the fact that the piston operates at both positive and negative applied bias suggests
% that the electrophoretic bias is not necessary for its functioning. Indeed, simulations
% indicate that entropy plays a more central role.
%
% LAMMPS.
%
% Discuss semiflexible bead-and-spring model DNA + repulsive LJ interactions to simulate
% excluded volume interactions.
%
% Each spherical ssDNA (dsDNA) bead represented 1 nt (ssDNA) or 5 bp (dsDNA), had a
% diameter of 1 nm (ssDNA) or 2.2 nm (dsDNA) and an average bond distance of 0.68 nm
% (ssDNA) or 1.7 nm (dsDNA).
%
% ClyA consisted of three connected open cylinders, with diameters 6 nm, 5.5 nm and 2.9 nm
% from the cis- to the trans-side, following the geometry reported in []
% %Bell, N. A.; Engst, C. R.; Ablay, M.; Divitini, G.; Ducati, C.; Liedl, T.; Keyser, U. F.
% %DNA Origami Nanopores. Nano Lett. 2012, 12 (1), 512−517. (4)
% The latter value is somewhat smaller than the reported one (3.3 nm), accounting for the
% electrostatic repulsion between DNA and the negatively-charged trans-entry.
% The lipid bi layer in which the pore is embedded is simulated by a reflective boundary at
% the lower entrance of the nanopore interacting only with the neutravidin beads.
%
% Neutravidin was modeled as a sphere of diameter 7 nm, attached to the two ends of the
% rotaxane. The motivation of this size was done by simulating the neutravidin bead
% together with the ClyA nanopore and fitting the size of the neutravidin bead so that it
% could be caputed in the lumen of the nanopore, as is seen in experiments.
%
%
% The bond strength is chosen to be, by transforming the equation for persistence length
% kSsDNA = persistenceSsDNA * boltzmann * temperature / lengthNt
% kDsDNA = persistenceDsDNA * boltzmann * temperature / lengthBp
% For the description of ssDNA and dsDNA, with persistence lengths 2.2 nm and 45 nm,
% respectively.
%
% $k_{bond} = l_p * k_{bt} * T / \langle a \rangle$
% Kuhn segments for ssDNA en dsDNA are different!
%
% Gaussian probably distribution of the beads, i.e. the beads show ideal chain behaviour.
% Equipartition theorem.
%
% $k_{bond} = 3 k_b T / \langle a \rangle$
%
% Langevin integrator is used As is common in simula- tions of coarse-grained models. As is
% common practice in MD simulations, the diffusion coefficient of the oxDNA strand is
% chosen larger then the value of physical DNA. This is done to speed up the simulations,
% while ensuring its physical accuracy.
%
% Discuss limits of the model. Limited accuracy of the CG model since it does not caputre
% the full structure of DNA accurately. For example the double helix structure is note
% captured. Another consequence is that the DNA hybridisation can not be simulated using
% this model, since both ssDNA nucleotides and dsDNA basepairs are represented by simple
% beads. To further analyse and understand the operation cycle of the nanopiston a more
% accurate CG model is needed.
