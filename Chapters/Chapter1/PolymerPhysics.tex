\section{Polymer Physics}
The theory of polymer physics logically starts with defining the notion of a polymer, for
simplicity we will limit the discussion to linear polymers. A
polymer is made up of building blocks called monomers, linked together to from a chain.
The configuration of this chain is determined by the position vector of each monomer,
$\{\boldsymbol{r}_0, \boldsymbol{r}_1, \dots, \boldsymbol{r}_N\}$. The link between each
consecutive pair of monomers is called a bond and defined by bond vectors,
$\boldsymbol{u}_i = \boldsymbol{r}_i - \boldsymbol{r}_{i-1}$. During this discussion we
will assume these bonds to be inextensible, i.e. having a fixed
segment length $|\boldsymbol{u}_i| = a$.

Many different polymer models have been developed to accurately represent a physical
polymer using this rather simple definition. The most simple models are called the ideal
flexible polymer, here a probably distribution is assigned to each bond vector.
\begin{equation}
    g(\boldsymbol{u}) = \frac{1}{4 \pi a}
    \delta(|\boldsymbol{u}| - a)\\
\end{equation}


\begin{equation}
    g(\boldsymbol{u}) = (\frac{3}{2 \pi a^2}) e^{\frac{-3
    \boldsymbol{u}^2}{2 a^2}}
\end{equation}

The Kratky-Porod, or discrete worm like chain, model is an example of a mathematically
simple model that results in a surprisingly realistic description of a polymer. This
model is an ideal semi-flexible polymer, which means that the energetic cost of bending a
polymer is taken into account. Mathematically this is done introducing a bending
stiffness between consecutive bonds, each polymer configuration is assigned an energy
using the equation,
\begin{equation}
    E_{WLC}= -\kappa \sum_{i=1}^{N} \boldsymbol{\hat{u}_i} \cdot
    \boldsymbol{\hat{u}}_{i+1}
    = -\kappa
    \sum_{i=1}^{N} \cos\theta_i.
\end{equation}

\begin{equation}
\begin{aligned}
    \label{210}
    Z_{\mathrm{WLC}}(N, T)
    &= \int_{0}^{\pi}\dots \int_{0}^{\pi} d \theta_1 \dots d \theta_N \sin \theta_1 \dots
    \sin \theta_N\ e^{\beta \kappa \sum_{i=1}^{N-1} \cos\theta_i}\\
    &= \left[\int_{0}^{\pi} d \theta \sin \theta e^{\beta \kappa \cos
    \theta}\right]^{N}\\
    &= \left[Z_{\mathrm{WLC}}(1, T)\right]^{N}
\end{aligned}
\end{equation}

\begin{equation}
    Z_{\mathrm{WLC}}(1, T)=\int_{0}^{\pi} d \theta e^{\beta \kappa \theta}=\frac{2 \sinh(\beta
    \kappa)}{\beta \kappa}
\end{equation}
\begin{equation}
    \left\langle\cos \theta_{i+1}\right\rangle
    =\frac{\partial \log Z_{\mathrm{WLC}}(1, T)}{\beta \partial \kappa},
\end{equation}
correlation between consecutive bond vectors.
\begin{equation}
    \left\langle\boldsymbol{u}_{i} \cdot \boldsymbol{u}_{i+1}\right\rangle
    = \left\langle\cos \theta_{i+1}\right\rangle
    = \frac{1}{\tanh(\beta \kappa)} - \frac{1}{\beta \kappa}.
\end{equation}
the limit of $\beta \kappa \gg 1$
\begin{equation}
    \langle\cos \theta\rangle \approx 1-\frac{1}{\beta \kappa}.
\end{equation}
This condition is true for low temperatures or large stiffness between neighbouring bonds
(large \kappa).

$\boldsymbol{\vec{u}}_{n-1} = \boldsymbol{\vec{u}}_{n} \cos(\theta_{n-1}) +
\boldsymbol{\vec{u}}_{n}^{\perp} \sin(\theta_{n})$


\begin{equation}
\begin{aligned}
    \left\langle\boldsymbol{\hat{u}}_{i} \cdot \boldsymbol{\hat{u}}_{i+m}\right\rangle
    &=\left\langle\boldsymbol{\hat{u}}_{i} \cdot
        \boldsymbol{\hat{u}}_{i+m-1}\right\rangle\left\langle\cos
    \theta_{n}\right\rangle = \dots =\langle\cos \theta\rangle^{m}
    % &=\exp \bigg{[} -\frac{n a}{l_{b}} \bigg{]},
\end{aligned}
\end{equation}

\begin{equation}
    \left\langle\boldsymbol{\hat{u}}_{i} \cdot \boldsymbol{\hat{u}}_{i+m}\right\rangle =
    \exp(m\ \log(1 - \frac{1}{\beta \kappa})) \approx \exp(-na/l_p)
\end{equation}

\begin{equation}
l_b \equiv \frac{a \kappa}{k_{b} T}
\end{equation}

The end-to-end vector $\boldsymbol{R}$ in the continuum limit. For example can be
rewritten using the arc-length parameter $s$ where $0 \leq s \leq L$ and $L = Na$ as:
\begin{equation}
    \label{hoi}
    \langle\widehat{u}(q) \cdot \widehat{u}(q+s)\rangle= e^{-s / l_{\mathrm{p}}}.
\end{equation}
The continuum version of the end-to-end vector
\begin{equation}
    \boldsymbol{R}=\int_{0}^{L} \widehat{u}(s) d s.
\end{equation}
\begin{equation}
\begin{aligned}
    \left\langle\boldsymbol{R}^{2}\right\rangle
    &= \int_{0}^{L} d s d s^{\prime}\left\langle\widehat{u}(s) \cdot
  \widehat{t}\left(s^{\prime}\right)\right\rangle \\
    &= 2 l_{\mathrm{b}} L\left\{1-\frac{l_{\mathrm{b}}}{L}\left(1-e^{-L /
l_{\mathrm{b}}}\right)\right\}.
\end{aligned}
\end{equation}

Explain here two limiting cases of above expression. This leads to a interpretation of
what the persistence length means intuitively. -> length over which the correlations
between bond boldsymboltors are lost.
