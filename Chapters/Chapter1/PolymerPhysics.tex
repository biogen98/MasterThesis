\section{Polymer Physics}

What is a polymer?\\

\begin{equation}
    E_{WLC}= -K \sum_{i=1}^{N-1} \hat u_i \cdot \hat u_{i+1} = -K \sum_{i=1}^{N-1} \cos\theta_i,
\end{equation}

\begin{equation}
\begin{aligned}
    \label{210}
    Z_{\mathrm{WLC}}(N, T)
    &= \int d \theta_{1} d \theta_{2} \ldots d \theta_{N} \sin \theta_{1} \sin \theta_{2} \ldots \sin \theta_{N} \exp \bigg{[} \beta K \sum_{i=1}^{N} \cos \theta_{i} \bigg{]} \\
    &= \left[\int_{0}^{\pi} d \theta \sin \theta e^{\beta K \cos \theta}\right]^{N}
\end{aligned}
\end{equation}

\begin{equation}
    Z_{\mathrm{WLC}}(1, T)=\int_{0}^{\pi} d \theta e^{\beta K \theta}=\frac{2 \sinh(\beta
    K)}{\beta K}
\end{equation}
Using the relation:
\begin{equation}
    \left\langle\cos \theta_{i+1}\right\rangle
    =\frac{\partial \log Z_{\mathrm{WLC}}(1, T)}{\beta \partial K},
\end{equation}
one can calculate the mean dot product between two consecutive bonds:
\begin{equation}
    \left\langle\vec{u}_{i} \cdot \vec{u}_{i+1}\right\rangle
    = \left\langle\cos \theta_{i+1}\right\rangle
    = \frac{1}{\tanh(\beta K)} - \frac{1}{\beta K}.
\end{equation}
Which in the limit of $\beta K \gg 1$ leads to:
\begin{equation}
    \langle\cos \theta\rangle \approx 1-\frac{1}{\beta K}.
\end{equation}
This condition is true for low temperatures or large stiffness between neighbouring bonds (large K). In the following equations the persistence length $l_b$ is mathematically introduced:
\begin{equation}
\begin{aligned}
    \left\langle\vec{u}_{1} \cdot \vec{u}_{n+1}\right\rangle
    &=\left\langle\vec{u}_{1} \cdot \vec{u}_{n}\right\rangle\left\langle\cos
    \theta_{n}\right\rangle
    &= a^{2}\langle\cos \theta\rangle^{n}
    &=a^{2} \exp \bigg{[} -\frac{n a}{l_{b}} \bigg{]},
\end{aligned}
\end{equation}
where $l_b \equiv \frac{a K}{k_{b} T}$ and $a$ is the constant bond length for the discrete WLC model.

The bending persistence length $l_b$ can be related to the end-to-end vector $\vec{R}$ in
the continuum limit. For example can be rewritten using the arc-length parameter $s$
where $0 \leq s \leq L$ and $L = Na$ as:
\begin{equation}
    \label{hoi}
    \langle\widehat{u}(q) \cdot \widehat{u}(q+s)\rangle= e^{-s / l_{\mathrm{p}}}.
\end{equation}
The continuum version of the end-to-end vector is:
\begin{equation}
    \vec{R}=\int_{0}^{L} \widehat{u}(s) d s.
\end{equation}
Taking the square of this expression and averaging over it, we obtain the following:
\begin{equation}
\begin{aligned}
    \left\langle\vec{R}^{2}\right\rangle
    &= \int_{0}^{L} d s d s^{\prime}\left\langle\widehat{u}(s) \cdot
  \widehat{t}\left(s^{\prime}\right)\right\rangle \\
    &= 2 l_{\mathrm{b}} L\left\{1-\frac{l_{\mathrm{b}}}{L}\left(1-e^{-L /
l_{\mathrm{b}}}\right)\right\}.
\end{aligned}
\end{equation}

Explain here two limiting cases of above expression. This leads to a interpretation of
what the persistence length means intuitively. -> length over which the correlations
between bond vectors are lost.
