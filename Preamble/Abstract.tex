\addcontentsline{toc}{chapter}{Abstract}
\chapter*{Abstract}

Autonomous molecular machines are ubiquitous in the machinery of life, driving molecular
processes at the nanoscale. Inspired by these biological machines, scientists develop
synthetic devices performing specialised operations at this length scale. In this
thesis we study a specific molecular machine designed by Bayoumi et al.\cite{Bayoumi21},
which is composed of a DNA-neutravidin piston trapped inside a ClyA nanopore.

Using the free energy of DNA hybridisation, this molecular machine is able to perform
autonomous and active transport of DNA cargo both following or against an
external bias. During each operating cycle of the nanopiston a DNA cargo
is transported from the cis- to the trans-side of the membrane in which the nanopore is
embedded.

Due to the length scale associated with molecular machines, studying the mechanisms
driving the operation cycle is experimentally challenging. During this thesis we aim
to shed light on the operating principles of the nanopiston by using molecular dynamics
simulations. Motivated by the computational cost of classical all-atom simulations, a
coarse-grained model of the DNA nanopiston was designed. In this work the popular OxDNA
model was used to simulate the DNA strands, where the other components of the complex
were modelled using Lennard-Jones beads.

Entropic interactions between the DNA piston and the nanopore are thought to
play an important role in facilitating the DNA transport. The contribution of
these effects are studied in a variation of the original molecular machine, where the
fraction of double and single stranded DNA is varied. In our simulations we
observe that an equilibrium is found between competing entropic forces. The large double
stranded DNA is kept outside of the pore's constriction, while the flexible single
stranded DNA strives to maximize its available configurational space.

Having explored the entropic effects related to the confinement of DNA strands in the
nanopore, next the typical conformations of the stable states in the operating cycle
are simulated. Here the importance of the entropic interactions promoting the
hybridisation reactions clearly come to light. The entropic penalty of confining the
flexible single stranded DNA components of the piston in the nanopore enables the
continuation of the operation cycle.

\newpage

In an attempt to gain an overall understanding of the transition pathways driving the
molecular device, the hybridisation reactions are simulated using our coarse-grained
model. Due to the inherent difficulty of simulating DNA hybridisation an advanced
sampling method called forward flux sampling is needed to study these phenomena.
While performing these simulations the limits of our coarse-grained model are
encountered. The compliance of the biological nanopore is found to be essential in
facilitating the hybridisation pathways, but is not incorporated in our current model.

\cleardoublepage
