\section{Conformational Fluctuations of Rotaxanes-ds and -ss}
% \vspace{-0.2cm}

Having gained insight into the entropic interactions between the DNA and the
nanopore in the mixed rotaxanes, now the stable intermediate states of the nanopiston's
operation cycle are
studied. Both the rotaxane-ss and -ds are composed of stiff dsDNA and flexible ssDNA
parts, which characterise the conformational fluctuations by their entropic
interactions.
The two rotaxanes types are simulated in a ClyA nanopore, using the predefined
coarse-grained
model, in absence of an external bias, i.e.  $0\ mV$. The results are presented in Figure
\ref{fig:conform}.

Analysing the conformational fluctuations of the rotaxane-ds, we observe from the blue
histogram that the ssDNA
overhang remains predominantly outside of the pore. This effect can be explained by
taking
into account the entropic cost of capturing the flexible strand of ssDNA into the
constriction of the pore. Since the geometry of the rotaxane prohibits the overhang from
reaching the cis-side of the pore, the overhang can only freely fluctuate on the
trans-side of the pore. Throughout the simulation this entropic force keeps the overhang
in the trans-reservoir and thereby placing the cis-stopper close to the entrance
of the pore. This entropic interaction plays an important role in the operation of the
nanopiston. Even when an external voltage difference induces an upward electrophoretic
force on the rotaxane, the competing entropic force keeps the overhang outside of the
constriction and thereby exposing it for hybridisation with a fuel strand. This also
explains the halting of the piston cycle at high voltages. In this case the entropic
force is overcome by the high electrophoretic force, sequestering the overhang inside
of the pore inhibiting the binding of fuel strands.

In the same figure the results for the rotaxane-ss are presented. We observe that the
histograms are shifted upwards, indicating an upward entropic force arising from the high
flexibility of the long ssDNA strand. This force originates from the increase in
configurational microstates available to the rotaxane-ss, when the ssDNA strand is
allowed to freely fluctuate in the
cis-reservoir. The flexibility of the ssDNA strand overcomes the entropic penalty of
confining the dsDNA into the constriction of the pore. Capturing the dsDNA fraction of
the
rotaxane-ss inside of the pore promotes the
operation cycle. In this case a longer ssDNA strand is exposed to the cis-reservoir,
better facilitating the hybridisation with cargo strands. This can be seen in the
positional histogram of the cis-protein stopper, where the large fluctuations of the
ssDNA strand allow the stopper to venture far away from the nanopore.

These results explain the functional importance of the entropic interactions in the
nanopiston's operating cycle. Comparing our findings with the simulation results by
Bayoumi et al., we see that both models are in reasonable accordance. However, in our
simulations the interface of the rotaxane-ss is observed entering inside the pore's
constriction, while in the bead-and-spring model this behaviour is not observed. This
difference can be attributed to the more accurate simulating of ssDNA by OxDNA, mainly
arising from the more precise parametrisation of the model and the ability for
consecutive bases to unstack.

\begin{figure}[ht!]
\begin{center}
  \includegraphics[width=0.95\textwidth]{Figures/image.png}
   \caption[Conformational fluctuations of the ss- and
ds-rotaxane.]{\linespread{0.5}{\small Conformational fluctuations of the rotaxane-ds
    (left) and -ss (right). Images of the atomistic structure for both
    rotaxane variants are shown in the center. On both sides the $X$-histograms for
    selected components of the rotaxanes are presented (see
    definition). The mean values of the
    $X$-coordinates are marked by the horizontal lines. From the results we conclude that
    the flexible ssDNA strands determine the fluctuations of the rotaxanes-ds and -ss.
    Central images were rendered using Blender.\cite{blender}}}
\label{fig:conform}
\end{center}
\end{figure}
