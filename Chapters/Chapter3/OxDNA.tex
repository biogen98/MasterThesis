\vspace{1cm}
\noindent Although the simulations discussed in the previous chapter provided an answer
to important questions regarding the piston's operation, some remarks need to be made.
Due
to the nature of this bead-and-spring model it does not allow for hybridisation reactions
to be simulated. Improving upon this model by using a more advanced coarse-grained
representation of DNA is therefore needed, providing also a more realistic description of
the conformational fluctuations. In this thesis we improved upon this model by using the
OxDNA coarse-grained model for DNA.\\


\section{OxDNA}


OxDNA is a coarse-grained model of DNA developed by Thomas E. Ouldridge et al. at the
University of Oxford.\cite{thomas2011a}\cite{Ouldridge2010} The central aim of the
project was to develop a coarse-grained model of DNA, that could be used in the design of
DNA technology. For the development of these technologies a model was needed that
accurately captured the structural, mechanical
and thermodynamical properties of DNA, while keeping the computational cost low.

The OxDNA model represents each nucleotide in the DNA strand as a rigid unit. Each rigid
nucleotide has three independent interaction sites, each capturing a different aspect of
the model. The interactions between these pseudo-atoms are compared to experimental
data to callibrate the interactions, characterising their approach as "top down"
coarse-graining. The interactions defined in the OxDNA model can be summarized as,

\begin{equation}
  \begin{aligned}
    V = \sum_{\text{nearest neighbours}} \bigg[ V_{\text{backbone}} + V_{\text{stack}} +
    V^{'}_{\text{exc}}\bigg]\\
    + \sum_{\text{other pairs}} \bigg[V_{\text{H.B.}} + V_{\text{cross stacking}} +
    V_{\text{exc}} + V_{\text{coax stack}}\bigg].
  \end{aligned}
\end{equation}

\begin{wrapfigure}[15]{r}{0.5\textwidth}
  \begin{center}
    \includegraphics[width=0.45\textwidth]{Figures/oxDNA_model.png}
  \end{center}
  \caption{Structure of the OxDNA model with the different interactions.
  Figure was taken form. \cite{Ouldridge2010}}
\end{wrapfigure}

The first interaction site is the hydrogen-bonding/base excluded volume site,
incorporating the hybridisation of complementary nucleotides into the model. The
hydrogen-bonding interactions are not fixed, allowing for OxDNA to simulate dsDNA, ssDNA
and their thermodynamic transitions.

The second is an excluded volume interaction site located at the backbone.
This site's main role is to simulate the covalent bonding between consecutive phosphate
groups. These permanent bonds provide structure to the ssDNA strands by forming the
backbone.

The last interaction site is again located at the base, where it provides a base stacking
interaction between consecutive nucleotides. The nucleotide stacking in DNA is crucial
for the formation of the characteristic helix structure. Using these stacking
interactions, this structure is implicitly imposed in the OxDNA model. This is in
contrast with the traditional approach, used in coarse grained-models like Martini
\cite{Souza2021} en
3SPN \cite{Freeman2011},
where the double helix structure is explicity constructed. This
implicit structure allows for the
unstacking of nucleotides, which especially in ssDNA is an important contribution to the
flexibility of the strand.

During the simulations of the DNA nanopiston, both the flexibility of the ssDNA strands
and the DNA thermodynamics play an important role. Since both aspect of DNA are
accurately captured by the OxDNA model, it provides a logical choice
for our simulations. The low number of degrees of freedom in the model allows us to
study computationally intensive simulations like DNA thermodynamics.
