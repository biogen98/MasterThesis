\section{Future Perspectives}

As is normal with research, understanding  question leads inevitably ,

1.
raise
opportunities that still remain to be ceased.

Martinin?
One of the conclusions of this thesis is that the compliancy of the Nanopore is
important. Especially the fluctuations int hte diameter of the constriction of the pore
is important to facilitate the hybridisation reactions driving the molecular machine.
Future research consists of incorporating a more accurate coarse grained model of the
Clya pore in the simulations of the nanopiston. A first proposition is modifying the
current model, where the constricton is incorporated in the langevin integrator,
parameterising the porebead interactions to reproduce experimental data. Another solution
to this problem is usign a protein coarse-grained model to use an more accurate cG model
to simulate the pore. Here the DNA could still be simulate using oxdna, or for coarse
grained models like martinie this could also ge used to simulate dna. IN the case of the
martinie forcefield the interacitons between the dna and the pore would be more accurete
using an all martini model, yet comparative studies show that oxdna better caputures the
dynamics of dna compared to martini dna.


Improved CG model of ClyA. Possible avenues. Building futher opon the relatively basic
model used in this work, but modifying the constricton of the pore by introducing bonds
between the pseudo atoms and adding them to the langevin integrator. Experimental results
can be used to parameterise these interactions, so that they accurately produce.

Using specific CG model for protein, GO CG models for  CG.

Using a CG model like martini that can both simulate the DNA and the nanpore.


Downside these two anvenues is that the computational cost of the simulations will
increase drastically.

2.
as new molecular machines are designed, combination with computational analysis helps
this.
