\section{Future Perspectives}

During this thesis we were able to deepen our understanding of the DNA
nanopiston. However, some important questions remained unanswered. While studying the
hybridisation reactions, which drive the molecular machine, we encountered the
limitations of
our coarse-grained model. Simulations indicated that the compliancy of the nanopore
is an essential component in facilitating these reactions. In our current model the ClyA
nanopore is represented as a static complex, not allowing the diameter fluctuations that
are needed for the hybridisation to take place.
Future research should attempt to incorporate a more accurate representation of the ClyA
nanopore in the coarse-grained model of the DNA nanopiston.

Various approaches can be taken to accomplish this improved accuracy.
A first proposition would be to expand further upon the existing model, by
incorporating the constriction of the pore in the Langevin integrator. Using experimental
data, we can parameterise the interactions between the constituent beads of the pore and
reproduce the diameter fluctuations of the constriction of ClyA. Another
possible solution could be to use an already parameterised coarse-grained model.
An example of such a model is the Martini force-field, which allows for accurate
simulations of
transmembrane proteins, like the ClyA pore. In this new model both OxDNA or the Martini
force-field could be integrated to simulate the DNA strand. These suggested improvements
would increase the accuracy of our model, probably enabling the simulation of a full
piston cycle.

Molecular dynamics simulations will proof to be a useful tool in the development of
new molecular machines. Future research is focused on designing molecular devices that
are able to not only transport DNA, but also other molecules through a membrane.
Eventually, the aim would be to implement these new molecular machines into a
sophisticated droplet-based device with emergent properties, i.e. a fully synthetic cell.

