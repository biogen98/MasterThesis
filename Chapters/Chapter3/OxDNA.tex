\section{OxDNA}

\begin{wrapfigure}{r}{0.45\textwidth}
  \begin{center}
    \includegraphics[width=0.42\textwidth]{Figures/oxDNA_model.png}
  \end{center}
  \caption{Structure of the OxDNA model with the different interactions.
          Figure was taken form [].}
\end{wrapfigure}

OxDNA is a coarse-grained model of DNA developed by Thomas E. Ouldridge et al. at Oxford
university. The central aim of the project was to develop a coarse-grained model of DNA
that could be used in the design of DNA technology. For the development of these
technologies a model was needed that accurately captured the structural, mechanical and
thermodynamical properties of DNA while keeping the computational cost low.

The OxDNA model represents each nucleotide in the DNA strand as a rigid unit. Each rigid
nucleotide has three independent interaction sites, each capturing a different aspect of
the model. The interactions between these pseudo atoms are next compared to experimental
data to tweak the interactions, characterising the approach as "top down"
coarse-graining. The interactions defined in the OxDNA model can be summarized as,

\begin{equation}
  \begin{aligned}
    V = \sum_{\text{nearest neighbours}} \bigg[ V_{\text{backbone}} + V_{\text{stack}} +
    V^{'}_{\text{exc}}\bigg]\\
    + \sum_{\text{other pairs}} \bigg[V_{\text{HB}} + V_{\text{cross stacking}} +
    V_{\text{exc}} + V_{\text{coax stack}}\bigg].
  \end{aligned}
\end{equation}

The first interaction site is the hydrogen-bonding/base excluded volume site,
incorporating the hybridisation of complementary nucleotides into the model. The
hydrogen-bonding interactions are not fixed, allowing for OxDNA to simulate dsDNA, ssDNA
and their thermodynamic transitions.

The second interaction site is an excluded volume interaction located at the backbone.
These interactions simulate the covalent bonding between consecutive phosphate groups
using the FENE (finitely extensible nonlinear elastic) bond type.

The last interaction site is again located at base where it provides a base stacking
interaction between consecutive nucleotides. The nucleotides stacking in DNA is crucial
for the formation of the characteristic helical structure. Using these stacking
interactions this structure is implicitly imposed in the model. Contrasting the common
approach of explicitly imposing the Double helix structure in other coarse grained-models
like 3SPN and Martini. This implicit structure allows for the unstacking of nucleotides,
which especially in ssDNA is an important contribution to the flexibility of the strand.

During the simulations of the DNA Nanopiston both the flexibility of the single stranded
DNA strands and the hybridisation reactions play an important role. Since both of these
aspect of DNA are accurately captured by the OxDNA model, it provides a logical choice
for our simulations. The low number of degrees of freedom in the model, allows us to
simulate computationally intensive simulations like DNA hybridisation.
