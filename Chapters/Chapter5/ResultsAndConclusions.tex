\noindent In this work the novel molecular machine devised by Bayoumi et
al.\cite{Bayoumi21} has been
studied using
molecular dynamics simulations. The primary objective of this research was to shed
light on the operating principles, facilitating the autonomous and active transport,
which
characterise this DNA nanopiston. In this chapter the obtained results are summarized,
after which recommendations for future research are made.

\section{Results \& Conclusions}

Understanding the underlying processes driving the operation of molecular machines is an
important corner stone in furthering their development. As a result of the scale and
complexity of these structures, performing  experimental studies aiming to shed light on
these interactions has been proven to be challenging. In this work we aimed to provide an
insight into the operation of the
DNA nanopiston using molecular dynamics simulations.

This thesis built further upon simulations presented in the original paper, where a
coarse-grained model of the DNA nanopiston was developed based on both theoretical and
experimental considerations. The central component of this model was the DNA rotaxane,
which was simulated using a bead-and-spring model. To increase the scope of the research
we designed a new coarse-grained model of the DNA nanopiston, utilising a more accurate
representation of DNA, namely OxDNA. This new model yielded a more realistic description
of DNA, at the same time enabling us to simulate DNA hybridisation reactions.

The entropic penalty of confining single stranded DNA inside of the nanopore was presumed
to play an important role in the mechanisms driving the DNA piston. We studied these
entropic effects by simulating a specifically engineered class of rotaxanes, called the
mixed rotaxanes. These simulations indicated that two competing entropic
interactions were present in the rotaxane-pore complex. A first entropic force was found
to
arise from confining the large dsDNA strands of the rotaxane inside of the pore.
Competing with this force were the smaller but more flexible ssDNA strands, which
endeavoured to maximize their available configurational space by opposing the confinement
of the pore. As the lengths of these ssDNA strands were increased, this latter
interaction started to dominate.

After establishing this essential understanding of the entropic interactions, the two
stable states of the piston's operating cycle were simulated (i.e. rotaxane-ss and
rotaxane-ds). These simulations indicated that the sequestering of the rotaxane-ds
overhang was prevented by the entropic penalty of confining it in the pore.
The established entropic force enabled the hybridisation with fuel strands even opposing
an upward electrophoretic force, differentiating the DNA nanopiston as an active
transporter. Due to the rotaxane geometry this hybridisation reaction resulted in the
formation of rotaxane-ss in a low-entropy state. As a consequence the rotaxane-ss
spontaneously migrated in the trans-cis direction exposing the flexible ssDNA strand to
the cis-reservoir, enclosing the dsDNA fraction inside of the pore. This
upward motion played an important role in exposing the ssDNA fraction of rotaxane-ss to
the cis-reservoir, where it could hybridise with a cargo strand. Here we concluded that
the interplay of the hybridisation reactions and the entropic interactions collectively
drove the molecular machine.

In the last part of this thesis an attempt was made to study the ensemble of
hybridisation pathways driving the nanopiston. The study of
these thermodynamic transitions was complicated by the complex reaction kinetics and an
initial energy barrier. To overcome this difficulty a forward flux sampling algorithm
was used to analyse these transitions. From the performed simulations, we concluded that
the static representation of the ClyA pore in our coarse-grained model inhibited these
hybridisation reactions. The compliance of the pore constriction was found to be
essential in facilitating the hybridisation reactions. However, this was not yet
incorporated in our model.

The performed computational analysis of the DNA nanopiston confirmed previous research on
the operating principles of the piston cycle. The performed exploratory research
motivates further research towards the fundamental aspects of this nanopiston.
Eventually, this knowledge will aid the development of novel molecular devices, further
blurring the line between life and the artificial.
