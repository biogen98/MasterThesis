\section{Future Perspectives}

During this thesis we were able to deepen our understanding of the DNA
nanopiston, however some important questions remained unanswered. While studying the
hybridisation reactions which drive the molecular machine, we encountered the limits of
our coarse-grained model. Simulations indicated that the compliancy of the nanopore
is an integral component in facilitating these reactions. In our current model the ClyA
nanopore is represented as a static complex, not allowing the diameter fluctuations which
are needed for the hybridisation to take place.
Future research consists of incorporating a more accurate coarse-grained model of the
Clya pore in simulations of the nanopiston.\\

Various approaches can be taken to more accurately represent the nanopore in our model.
A first proposition is building further upon the current model, by incorporating the
constriction of the pore in the Langevin integrator. Using experimental data, we can
parametrise the interactions between the constituent beads of the pore and reproduce the
diameter fluctuations of the constriction of the Clya pore. Another possible
solution to this problem is using a different coarse-grained model to simulate the
pore. An example is the Martini force-field, which allows for accurate simulations of
transmembrane proteins like the Clya pore. In this new model both OxDNA or the Martini
forcefield could be used to simulate the DNA strand. Both these possible improvements
would increase the accuracy of our model, enabling the simulation of a full piston
cycle.\\

Molecular dynamics simulations will proof to be a useful tool in the development of
new molecular machines. Future research is focused on designing molecular devices that
are able to not only transport DNA but also other molecules through a membrane.
Eventually, the aim is to implement these new molecular machines into a sophisticated
droplet-based device with emergent properties, i.e. a fully synthetic cell.

