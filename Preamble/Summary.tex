\addcontentsline{toc}{chapter}{Vulgariserende Samenvatting}
\chapter*{Vulgariserende Samenvatting}

Als we door een biologische bril naar de natuur kijken zijn cellen de kleinste bouwstenen
van een organisme. Deze basiseenheid van de natuur is opgebouwd uit vele moleculaire
machines, die gezamenlijk het leven mogelijk maken. Denk hierbij aan de flagellen waarmee
bacteriën zich voortbewegen of de enzymen die ons erfelijk materiaal kopiëren bij de
celdeling. Wat deze biologische machines zo speciaal maakt, is hun kleine omvang. Vaak
zijn ze namelijk
niet veel groter dan enkele nanometers. Geïnspireerd door de vele taken die deze machines
kunnen volbrengen, onderzoeken wetenschappers hoe zij deze zelf kunnen namaken en
optimaliseren.

In dit onderzoek bestuderen wij een specifiek voorbeeld van een synthetische moleculaire
machine, die instaat is om DNA moleculen te transporteren. Deze machine kan vergeleken
worden met een zuiger uit een verbrandingsmotor, waarbij tijdens elke cyclus een molecule
wordt getransporteerd. Door de microscopische omvang van dit apparaat is het moeilijk
gebleken om de exacte mechanismen te bestuderen. Om toch een inzicht te verkrijgen in
deze
mechanismen kunnen computersimulaties worden gebruikt.

Door een computermodel van deze moleculaire machine te ontwerpen, kunnen we de
microscopische bewegingen van de individuele componenten volgen. Vanwege de complexiteit
van dit biologische apparaat is het niet mogelijk om elk atoom expliciet te modelleren.
Dit probleem wordt opgelost door gebruik te maken van een ‘coarse-grained’ model, waarbij
sommige moleculaire details worden genegeerd. In deze thesis dient onze computer als een
computationele microscoop om de kleinste processen te bestuderen.

Het uiteindelijke doel van deze studie is om beter te begrijpen hoe deze moleculaire
machine functioneert. Op deze manier kunnen onze simulaties bijdragen tot het ontwikkelen
van
nieuwere en betere versies.  De
vooruitgang in dit onderzoeksveld gaat zeer snel, waardoor wetenschappers voorzichtig
durven uit te kijken naar het ontwerpen en fabriceren van een volledig synthetische cel.


\cleardoublepage
\addcontentsline{toc}{chapter}{Summary in Laymans's Terms}
\chapter*{Summary in Layman's Terms}

If we look at nature from a biological perspective, cells are the smallest building
blocks of an organism. This basic unit of nature is made up of many molecular
machines that together make life possible. An example are the engines with which
bacteria move around or the machines that copy our genetic material during cell division.
What makes these devices so special is their small size, often they are not much
larger than a few nanometres. Inspired by the many tasks these small machines
can accomplish, scientists are busy investigating how they can design them in their lab.

In this work, we study a specific example of a synthetic molecular
machine, capable of transporting DNA strands. This machine can be compared to a piston
in an internal combustion engine, where a DNA strand is transported during each cycle.
Due to the microscopic size of this device, it is difficult to determine the exact
working mechanisms. To gain an insight in these mechanisms, computer simulations are
used.

By designing a computer model of this molecular machine, we can
analyse the microscopic movements of its individual components. Because of the complexity
of this biological device it is not possible to model every atom explicitly.
This problem is solved by using a 'coarse-grained' model, where
some molecular details are ignored. In this thesis, our computer serves the role of a
computational microscope by which we study these tiny processes.

The ultimate goal of this study is to better understand how this molecular machine
functions and guide the development of improved versions. The progress in this
research field is moving fast, compelling scientists to cautiously start thinking of
designing and fabricating a fully synthetic cell.

