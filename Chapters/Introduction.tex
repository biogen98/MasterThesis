\chapter{Introduction}

\epigraphfontsize{\small\itshape}
\todo{add citation}
\epigraph{“...if we were to name the most powerful assumption of all, which leads one on
and on in an attempt to understand life, it is that all things are made of atoms, and
that everything that living things do can be understood in terms of the jigglings and
wigglings of atoms.”}
{--- \textup{Richard P. Feynman}, The Feynman Lectures on Physics}

Here why simulation.
blik in iets dat we vaak niet zien.

\section{Deoxyribonucleic Acid}

\section{Polymer Physics}

\section{Computer Simulations}

\subsection{Molecular Dynamics}

\begin{center}
	\begin{tikzpicture}[
	squarednode/.style={rectangle, draw=blue!60, fill=blue!5, very thick, minimum width=50mm,
	minimum height=8mm},]
	%Nodes
	\node[squarednode]      (step1)                        {1};
	\node[squarednode]      (step2)       [below= 5mm of step1] {2};
	\node[squarednode]      (step3)       [below= 5mm of step2] {3};
	\node[squarednode]      (step4)       [below= 5mm of step3] {4};
	\node[squarednode]      (step5)       [below= 5mm of step4] {5};
	\node[squarednode]      (step6)       [below= 5mm of step5] {6};
	\node[squarednode]      (step7)       [below= 5mm of step6] {7};
	\node[squarednode]      (step8)       [below= 5mm of step7] {8};

	%Lines
	\draw[thick, ->] (step1.south) -- (step2.north);
	\draw[thick, ->] (step2.south) -- (step3.north);
	\draw[thick, ->] (step3.south) -- (step4.north);
	\draw[thick, ->] (step4.south) -- (step5.north);
	\draw[thick, ->] (step5.south) -- (step6.north);
	\draw[thick, ->] (step6.south) -- (step7.north);
	\draw[thick, ->] (step7.south) -- (step8.north);
	\draw[thick, ->] (step2.west)  -- +(-0.4,0) |-(step8.west);
	\end{tikzpicture}
\end{center}

- understanding many body - newtons algorithm
- insight in the dynamics -> simulate trajectories
- recent developments in techniques to simulate trajectories of rare event
-increased computational power
\subsection{Coarse Grained modelling}
