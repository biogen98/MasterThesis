\chapter{Methods}

\epigraph{Given for one instant an intelligence which could comprehend all
the forces by which nature is animated and the respective positions
of the beings which compose it, if moreover this intelligence were vast
enough to submit these data to analysis, it would embrace in the same
formula both the movements of the largest bodies in the universe and
those of the lightest atom; to it nothing would be uncertain, and the
future as the past would be present to its eyes.}
{--- \textup{Pierre-Simon Laplace}}

\section{Figures}
An example is Figure~\ref{Landslide}
\begin{figure}[h] % [h] bepaalt de plaats waar de figuur komt in de tekst
    \centering % figuur komt in het midden terecht
    \includegraphics[width=0.8\textwidth]{Figures/CoverPhoto.png}
    \caption{A landslide.}
    \label{Landslide}
\end{figure}

\newpage

\section{Tables}
An example is Table~\ref{Tabel1}
\begin{table}[h]
    \centering
    \begin{tabular}{|c|c|}
        \hline
       Model  &  Accuracy  \\ \hline
        regression & 90\%                \\ \hline
        random forests & 95\%           \\ \hline
    \end{tabular}
    \caption{A random table.}
    \label{Tabel1}
\end{table}

\newpage

\section{Equations}
Equations can be inserted in the text itself, working in the \textit{mathmode}(put text between \$-signs, for example $Y_i=\frac{1}{x}$). Or put them in the text as a numbered floating element (e.g.\ Equation~\eqref{Eq1}).
\begin{equation}\label{Eq1}
    y=\frac{1}{x}
\end{equation}

\begin{equation}
    y=\int_{a}^{b} x^2 dx
\end{equation}

\begin{equation}
    y=\sum_{i=1}^{n} x_i^2
\end{equation}

You can align the equations:

\begin{align}
    y &=\frac{1}{x} \\
    y &=\int_{a}^{b} x^2 dx \\
    y &=\sum_{i=1}^{n} x_i^2
\end{align}
