\section{Forward Flux sampling}

Computational methods are used to study a wide variety of phenomena, ranging from
large meteorological events to chemical reactions at the atomic scale. One class of
phenomena that is omnipresent in all these fields are the rare events. A rare event is an
event caused by stochastic fluctuations in the system, characterised by a large
gap in the time-scales of the duration of the events and their temporal spacing.
The infrequency of their occurrence in combination with their short duration, makes them
hard to study with both experimental and computational approaches.


Using this definition many natural processes can be classified as rare events, among
which the hybridisation and toehold displacement reactions studied in this thesis. Due to
the large temporal spacing of these rare events, brute-force molecular dynamics
simulations would simulate a lot of wait time between events. To effectively probe the
kinetics of these rare events, advanced sampling methods are needed.


A large ensemble of advanced sampling methods have been developed and can be largely
divided into two classes. The first class are the free energy methods, based upon
applying a biasing potential onto chosen collective variables. These potentials bias the
Hamiltonian of the system in such a way that rare parts of its configuration space are
explored. Notable examples of these methods are adaptive biasing force algorithm[.],
basis function sampling[.] and umbrella sampling[.]. %(SSAGES)


The second class of methods, known as path sampling methods, do not influence the
systems Hamiltonian, but rather interface directly with the simulations trajectories. The
transition path ensemble is usually sampled by perturbing an initial transition path or
partitioning the phase space in subregions. Examples of these methods are transition
path sampling [.] and forward flux sampling[.][.].  The latter will be used in our
hybridisation simulations, motivated by its relative simplicity.


Forward Flux Sampling (FFS) starts with identifying two local minima, $A$ and $B$, in the
energy landscape of our system, for which we want to sample the transition path ensemble.
Next an order parameter, $\lambda(x)$, is defined with the aim of partitioning the
phase space, $\Omega$, using a set of nonintersecting hypersurfaces. By design, we
choose this order parameter to be a function, $\lambda(.):\ $\Omega$\ \rightarrow
\mathcal{R}$, monotonically increasing from the initial state $A$ and too the final
state $B$.


Using this function the two local minima can
now be specified as $A := \{x: \lambda(x) < \lambda_A\} $ and $B := \{x: \lambda(x) \geq
\lambda_B\} $. The chosen levels of order, $\lambda_A$ and $\lambda_B$, construct the
interfaces separating the two local energy basins of our rare event from the rest of the
phase space. Finally this procedure can be done for a $N$-number of interfaces spanning
between $A$ and  $B$, we find
\begin{equation}
\lambda_A = \lambda_0 < \lambda_1< \dots < \lambda_{N-1} < \lambda_N = \lambda_B
\end{equation}
Note that this methods does not require an in depth knowledge of the energy landscape
between the two states, choice of order parameter will however heavily influence the
efficiency of the simulation.

When we study these rare events, we want to get to understand their kinetics or in other
words their reactions speed $k_{AB}$.

 \begin{equation}
    k_{AB} = \frac{\langle \Phi_{A,n} \rangle}{\langle h_{\mathcal{A}}\rangle} =
    \frac{\langle \Phi_{A,0} \rangle}{\langle h_{\mathcal{A}}\rangle}
    P(\lambda_n|\lambda_0)
 \end{equation}
 $\langle \Phi_{A,n} \rangle$ is the steady state flux from $A$ to  $B$ and
$\langle h_{\mathcal{A}}\rangle$ is the average fraction of time that  a trajectory
spends in the local minima basin $A$. In the above equation this steady state flux is
factorised into the flux over the interface corresponding to the first order lever,
$\lambda_0$ and the subsequent probability of reaching the final state from this
interface. Using the previously defined interfaces, we can now factorize the events'
probability into transition probabilities between the individual interfaces as

\begin{equation}
    P(\lambda_n|\lambda_0) = \prod_{i=0}^{n-1} P(\lambda_{i+1}|\lambda_i).
 \end{equation}

Estimate the transition probabilities can be done by shooting trajectories starting from
one interface to the next, while keeping track of the number of attempts. Since not the
entire energy landscape between the minima has to be crossed, measuring these small
transitions can be more easily done.


Note that this set-up allows for simulations of both equilibrium and out-of-equilibrium
systems, it does not require detailed balance like other sampling techniques.  Non
equilibrium systems are ubiquitous in soft matter physics, illustrating another strength
of the method.

\begin{figure}[ht]
\begin{center}
  \includegraphics[width=0.5\textwidth]{Figures/FFS.png}
  \caption{write caption[.]}
\end{center}
\end{figure}

Different variants on the FFS method have been devised, they differ in the approach by
which they calculate the probability $P(\lambda_n|\lambda_0)$. During this thesis I
choice is the “Branched Growth” (BG) algorithm [zie citation allen review], motivated by
the nicely implemented in recursive programming.

\begin{enumerate}
   \item Evaluate $\langle\phi_{A,0}\rangle$ + generate configs on $\lambda_0$
   \item fire $k_0$ trial runs.
   \item for each stored configuration firing runs at each interface
 $k_1$ to $lambda_2$ or back to  $
      lambda_1$
   \item Iterate until end is reached or no more configs, start over. Generate branching
      traj from one initial $\lambda_0$
   \item repeat 2 to 4 until succes.
\end{enumerate}
