\section{Computational setup}

The model used to study the DNA nanopiston, is largely based on the model
previously devised by Bayoumi et al. The difference main between the two models lays in
the coarse-grained model used to simulate the DNA strands. As discussed in Chapter 2, the
model of the previous simulated the DNA strands using a bead-spring approach, where now
we used oxDNA. The usages of this more sophisticated model gives a better represntation
of the dynamics of DNA strans, also allowing for accurate simulations of the
thermodynamic transitions in the DNA nanopiston.

The simulations were performed using the popular molecular dynamics simulator, LAMMPS[.].
Thanks to the Lammps implementation of oxDNA, developed by Henri et al[.],  this
simulator enabled us to simulate the interaction of oxDNA strands with externally defined
particles. The intial configurations of the simulations where generated using the
Moltemplate package[.], a general purpose molecule builder for LAMMPS.

The molecular dynamics simulation performed in this thesis utilised a langevin
thermostat, more precisely the $Dot-C$ langevin integrator also implemented by Henri et
al. This is a LAMMPS implementation of the “Langevin C” integrator developed Davidchack
et al. [.], falling in the class of  rigid-body Langevin-type integrators. This type of
thermostat separates the stochastic and deterministic parts of a Langevin thermostat to
efficiently take into account the extra degrees of freedom in the system, arising from
the non-spherical shape of the oxDNA beads. As is common practice in MD simulations, the
diffusion coefficient of the oxDNA strand is chosen larger then value of physical DNA.
This is done to speed up the simulations, while ensuring its physical accuracy.

The Model is used to study the more accurately study the conformational fluctuation of
the Rotaxane and develop understanding of the entropic interactions between the DNA and
the nanopore. Next the thermodynamic transitions in the operation cycle of the piston are
simulated using a forward flux sampling algorithm. The FFS algorithm is implemented as
Python script, performing the simulations by interfacing with the Python api of LAMMPS.
