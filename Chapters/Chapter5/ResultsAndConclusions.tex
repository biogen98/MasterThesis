In this work the novel molecular machine devised by Bayoumi et al. has been studied using
molecular dynamics simulations. The primary objective of this research was to shed
light on the operating principles facilitating the autonomous and active transport which
characterise this DNA nanopiston. In this chapter the obtained results are restated,
after which possible continuations of the research are explored.

\section{Results \& Conclusions}

The subject of investigation during this thesis was understanding the
operating using hybrid. and entropy. transport against external bias, hallmark of active
transport
understand using simulations since experiments give limited info.
improved on the model used in the original paper by more sophisticated representation of
dna.



mixed rotaxane -> competing entropic force, keeping large dsdna outside of constriction
while letting the ssdna fluctuate.



The simulations performed of the rotaxane-ss and rotaxane-ds helped to understand how to
individual interactions between the rotaxanes and the enclosing nanopore resulted in
facilitating active transport of the cargo. The entropic penalty of confining the
flexible toehold of rotaxane-ds was an important factor is this process. The resulting
entropic force prevents the sequestering of the toehold inside the
of the pore. This allows for the DNA nanopiston to transport
cargo even against an opposing electrophoric force, classifying it as an biological
active transporter. After the toehold displacement reaction, where a fuel strand
hybridises with the toehold, rotaxane-ss is formed. Due to the geometry of the nanopiston
complex, rotaxane-ss is formed with the ssDNA strand enclosed in the pore. This
low-entropy state forces the rotaxane-ss to migrate in this trans-cis direction exposing
the flexible ssDNA strand to the cis-basin and even enclosing the dsDNA fraction inside
of the pore. This piston movement is refered to as the recovery stroke of our molecular
engine, which plays an important role in exposing the ssDNA strand of rotaxane-ss to the
cis-reservoir where hybridises with the cargo strand. Here we concluded that the
interplay of the hybridisation reactions providing the free energy to facilitate the DNA
transport with the entropic interactions of the rotaxane with the enclosing pore work
together to drive the molecular machine.


hybridisation using ffs. Not able to Simulate using this model. did derive a conclusino
that compliancy of nanopore is important. observed in experimental data but not
incorporated in our model.

As is often the case with exploratory research hurdles in the experiments are encoutered,
which provide interesting avenues for further exploration.

in this thesis better understand the mechanisms driving the dna nanopiston.


% Using DNA strands as fuel, this DNA nanopiston is able to perform unidirectional
% molecular motion facilitated by the free energy of the DNA hybridization reactions.
%
%
% is obtained by focusing the free energy of DNA
% hybridization inside the nanopore cylinder and by exploiting the different entropy
% ofnanoscale confinement
%
% which is based upon a new operating principle
%
% During each cycle of the nanopiston, a ssDNA molecule is transported from the cis to the
% trans compartment through the nanopore. This device can operate at positive and negative
% applied potentials, indicating that the free energy ofDNA hybridization allows moving DNA
% against an external bias force, which is the hallmark ofbiological active transporters.
%
%
% window into analysis the piston was measerd pore current, limiting analysis. computer
% simulations where needed.
%
% Simulations where perfumed. Using coarse grained model for smutting, previously using
% bead-and-spring model. in this these we improved upon this model by using a more accurate
% OxDNA coarse-grained model. Compared to the bead and spring model provides the
% possibility to study the hybridisation reaction driving the operating cycle of the
% molecular machine, also give a more accurate description the DNA due to the better
% parametrisation. Here especially the ssDNA simulations where improved, because oxdna
% captures the unstacking of consecutive bases, regularly seen in ssDNA.
%
% mixed rotaxane varying compositions
%
% conformal fluctuations of rotaxane. --> ssdna is different.
%
% dna hybridisation.
%
% exploratory nature of research.
%
% this thesis succeeded in shedidng light on these aspects. Using the more accurate coarse
% granied model, largly the same as the bead-springmodel but some deviations where
% ofbserved. limitation of the model was observed, concluding that the compliancy of the
% pore is essential to the opertion of the piston.
%
% important research in the devlopment of new synthetic molecular machines,

% blurring the lines between life and the artificial.

% autonomous operation ubiquitous to the machinery of life
